%---------change this for every latex homework
\def\yourid{mst3k}
\def\collabs{list collaborators's computing IDs}
\def\sources{Cormen, et al, Introduction to Algorithms.  \emph{(add others here)}}
% -----------------------------------------------------
\def\duedate{Friday, April 22, 2022 at 11:30 pm }
\def\duelocation{via Gradescope}
\def\htype{Adv}
\def\hunit{C}
\def\hnumber{}
\def\course{{cs4102 - algorithms - spring 2022}}%------
%-------------------------------------
%-------------------------------------

\documentclass[10pt]{article}
\usepackage[colorlinks,urlcolor=blue]{hyperref}
\usepackage[osf]{mathpazo}
\usepackage{amsmath,amsfonts,amssymb, graphicx}
\usepackage{latexsym}
\usepackage[top=1in,bottom=1.4in,left=1.25in,right=1.25in,centering,letterpaper]{geometry}
\usepackage{color}
\definecolor{mdb}{rgb}{0.1,0.6,0.4} 
\definecolor{cit}{rgb}{0.05,0.2,0.45} 
\pagestyle{myheadings}
\markboth{\yourid}{\yourid}
\usepackage{clrscode}
\usepackage{tabularx}
\newcolumntype{Y}{>{\centering\arraybackslash}X}

\newenvironment{proof}{\par\noindent{\it Proof.}\hspace*{1em}}{$\Box$\bigskip}
\newcommand{\handout}{
   \renewcommand{\thepage}{Unit \hunit: \htype~Homework \hnumber~-~\arabic{page}}
   \noindent
   \begin{center}
      \vbox{
    \hbox to \columnwidth {\sc{\course} \hfill}
    \vspace{-2mm}
       \hbox to \columnwidth {\sc due \MakeLowercase{\duedate} \duelocation\hfill {\Huge\color{mdb}\hunit\hnumber{\Large\MakeLowercase{\htype}}(\yourid)}}
      }
   \end{center}
   \vspace*{1mm}
   \hrule
   \vspace*{1mm}
    {\footnotesize \textbf{Collaboration Policy:} You are encouraged to collaborate with up to 3 other students, but all work submitted must be your own {\em independently} written solution. List the computing ids of all of your collaborators in the \texttt{collabs} command at the top of the tex file. Do not share written notes, documents (including Google docs, Overleaf docs, discussion notes, PDFs), or code.  Do not seek published or online solutions for any assignments. If you use any published or online resources (which may not include solutions) when completing this assignment, be sure to cite by naming the book etc.\ or listing a website's URL. Do not submit a solution that you are unable to explain orally to a member of the course staff. Any solutions that share similar text/code will be considered in breach of this policy. Please refer to the syllabus for a complete description of the collaboration policy.
   \vspace*{1mm}
    \hrule
    \vspace*{2mm}
    \noindent
    \textbf{Collaborators}: \collabs\\
    \textbf{Sources}: \sources}
    \vspace*{2mm}
    \hrule
    \vskip 2em
}

\newcommand{\solution}[1]{\color{blue}\hfill\break\noindent\textbf{Solution:} #1\color{black}}
%\newcommand{\altsolution}[1]{\color{blue}\hfill\break\noindent\textbf{Solution (Alternative):} #1\color{black}}

%\newcommand{\solution}[1]{}
\newcommand{\altsolution}[1]{}

\newcommand{\bit}[1]{\{0,1\}^{ #1 }}
%\dontprintsemicolon
%\linesnumbered
\newtheorem{problem}{\sc\color{cit}problem}
\newtheorem{practice}{\sc\color{cit}practice}
\newtheorem{lemma}{Lemma}
\newtheorem{definition}{Definition}
\newtheorem{theorem}{Theorem}

\newcommand{\Z}{\mathbb{Z}} % This might be useful for Integers!

\begin{document}
\thispagestyle{empty}
\handout

%----Begin your modifications here



\begin{problem}Backpacking\end{problem}
You are going on a backpacking trip through Shenandoah National park with your friend. You two have just completed the packing list, and you need to bring $n$ items in total, with the weights of the items given by $W=(w_1, w_2, \ldots, w_n)$. You need to divide the items between the two of you such that the difference in weights is as small as possible. The total number of items that each of you must carry should differ by at most $1$. Use dynamic programming to devise such an algorithm, and prove its correctness and running time.  You may assume that $M$ is the maximum weight of all the items (i.e., $\forall i, \ w_i \leq M$). The running time of your algorithm should be a polynomial function of $n$ and $M$. The output should be the list of items that each will carry and the difference in weight.


\begin{problem}Course Scheduling\end{problem}
The university registrar needs your help in assigning classrooms to courses
for the fall semester. You are given a list of $n$ courses, and for each course $1 \le i \le n$,
you have its start time $s_i$ and end time $e_i$. Give an $O(n \log n)$
algorithm that finds an assignment of
courses to classrooms which minimizes the {\em total number} of classrooms required. Each classroom
can be used for at most one course at any given time. Prove both the correctness and running time
of your algorithm.


\begin{problem}Ubering in Florin\end{problem}
After the adventures with Westley and Buttercup in \emph{The Princess Bride}, Inigo decides to turn down the "Dread Pirate Roberts" title and to instead moonlight as the sole Uber driver in Florin.  He usually works after large kingdom-wide festivities at the castle and takes everyone home after the final dance.  Unfortunately, since his horse can only carry one person at a time, he must take each guest home and then return to the castle to pick up the next guest.

There are $n$ guests at the party, guests $1, 2, ..., n$.  Since it's a small kingdom, Inigo knows the destinations of each party guest, $d_1, d_2, ..., d_n$ respectively, and he knows the distance to each guest's destination.  He knows that it will take $t_i$ time to take guest $i$ home and return for the next guest.  Some guests, however, are very generous and will leave bigger tips than others; let $T_i$ be the tip Inigo will receive from guest $i$ when they are safely at home.  Assume that guests are willing to wait after the party for Inigo, and that he can take guests home in any order he wants.  Based on the order he chooses to fulfill the Uber requests, let $D_i$ be the time he returns from dropping off guest $i$.  Devise a greedy algorithm that helps Inigo pick an Uber schedule that minimizes the quantity:
\[ \sum_{i=1}^{n} T_i \cdot D_i.\]
In other words, he wants to take the large tippers the fastest, but also want to take into consideration the travel time for each guest.  Prove the correctness of your algorithm.  (Hint: think about a property that is true about an optimal solution.)


\begin{problem} Gradescope Submission \end{problem}
Submit a version of this \verb|.tex| file to Gradescope with your solutions added, along with the compiled PDF.  You should only submit your \verb|.pdf| and \verb|.tex| files.


\end{document}
